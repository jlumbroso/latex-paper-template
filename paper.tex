% Load configuration for submission mode
% This allows switching between anonymous submission and camera-ready versions
\IfFileExists{paper-config.tex}{%
    % Paper Configuration File
% This file controls the submission mode for the paper
% Edit this setting to switch between anonymous and camera-ready versions

% Submission mode toggle
% Set to true for anonymous submission, false for camera-ready version
\newif\ifanonymous
\anonymoustrue  % Currently set for anonymous submission

% For camera-ready version, change the line above to:
% \anonymousfalse%
}{%
    % Default to camera-ready if config file doesn't exist
    \newif\ifanonymous
    \anonymousfalse  % Camera-ready mode by default
}

% ACM template - change this to match your target conference/journal
\documentclass[letterpaper]{acmart}

%% ============================================
%% PACKAGES
%% ============================================

% Core packages
\usepackage{natbib}
\usepackage{graphicx}
\usepackage{booktabs}
\usepackage[hyphens]{url}
\urlstyle{rm}

% Typography
\usepackage{times}
\usepackage{helvet}
\usepackage{courier}
\frenchspacing

% Additional useful packages
\usepackage{algorithm}
\usepackage{algorithmic}
\usepackage{amssymb}      % For math symbols
\usepackage{pifont}       % For special symbols
\usepackage{subfig}       % For subfigures
\usepackage{float}        % For better figure placement
\usepackage{comment}      % For conditional content blocks

% TODO notes package - helpful for drafting
% Use \todo{...} for margin notes, \todo[inline]{...} for inline notes
% Remove or compile with 'final' option before submission
\usepackage{todonotes}

% Caption formatting
\usepackage{caption}
\captionsetup{font=small,labelfont=bf,textfont=normal}

% Page setup
\setlength{\pdfpagewidth}{8.5in}
\setlength{\pdfpageheight}{11in}
\setcounter{secnumdepth}{0}

%% ============================================
%% ANONYMIZATION HELPERS
%% ============================================

% Generic helper macro for inline anonymization
% Usage: \anontext{anonymous version}{named version}
\ifanonymous
    \newcommand{\anontext}[2]{#1}
\else
    \newcommand{\anontext}[2]{#2}
\fi

% Example custom macros for commonly-used terms
% Customize these for your paper, or remove if not needed
\ifanonymous
    \newcommand{\OurInstitution}{[Institution Anonymized]}
    \newcommand{\OurDepartment}{[Department Anonymized]}
    \newcommand{\OurLab}{[Lab Name Anonymized]}
\else
    \newcommand{\OurInstitution}{Example University}
    \newcommand{\OurDepartment}{Department of Computer Science}
    \newcommand{\OurLab}{AI Research Lab}
\fi

% Common abbreviation
\def\etal{\textit{et al.}}

%% ============================================
%% ACKNOWLEDGMENTS COMPATIBILITY
%% ============================================

% Ensure acks environment exists for compatibility across templates
% Automatically suppresses acknowledgments in anonymous mode
\ifx\acks\undefined
    \ifanonymous
        % In anonymous mode, suppress acknowledgments entirely
        \excludecomment{acks}
    \else
        % In camera-ready mode, create acknowledgments section
        \newenvironment{acks}{\section*{Acknowledgments}}{}
    \fi
\fi

%% ============================================
%% METADATA
%% ============================================

% PDF metadata
\pdfinfo{
/TemplateVersion (2026.1)
}

% Title - Update this with your paper title
\title{Example Paper Title: Demonstrating the Template}

% Author information - Update with your details
\ifanonymous
    \author{Anonymous Submission}
\else
    \author{
        First Author\\
        Example University\\
        Department of Computer Science\\
        City, State, Country\\
        first.author@example.edu
        \and
        Second Author\\
        Another University\\
        Department of Engineering\\
        City, State, Country\\
        second.author@example.edu
    }
\fi

%% ============================================
%% DOCUMENT BEGINS
%% ============================================

\begin{document}

% NOTE: For ACM template, abstract must come BEFORE \maketitle
\begin{abstract}
This is an example abstract demonstrating the format and typical structure.
The abstract should concisely summarize the problem, approach, and key contributions of the paper.
It typically ranges from 150-250 words and should be self-contained, allowing readers to understand the paper's scope without reading further.
Replace this text with your actual abstract.
\end{abstract}

\maketitle

%% ============================================
\section{Introduction}
\label{sec:intro}

This is a fixture introduction demonstrating typical paper structure.
A good introduction typically includes several key elements that guide the reader into your work.

\paragraph{Motivation and Context.}
Start by establishing why your research area matters.
What is the broader context?
Why should readers care about this problem?
For example, recent advances in machine learning have enabled new applications~\cite{Smith2023}, but significant challenges remain in scalability and efficiency.

\paragraph{The Problem.}
Clearly articulate the specific problem you're addressing.
What gap exists in current knowledge or practice?
Despite progress in neural network architectures~\cite{Chen2024,Davis2022}, existing approaches struggle with computational efficiency when deployed at scale.

\paragraph{Our Approach.}
Briefly describe your solution or contribution at a high level.
What is your key insight or innovation?
We propose a novel method that leverages adaptive optimization to reduce training time while maintaining accuracy.

\paragraph{Contributions.}
Enumerate your specific contributions in a clear, bulleted format.
This helps reviewers quickly understand what is novel about your work:
\begin{itemize}
    \item A new algorithm for efficient neural network training that reduces computation by 40\%.
    \item Theoretical analysis demonstrating convergence guarantees under realistic assumptions.
    \item Empirical evaluation on five benchmark datasets showing state-of-the-art performance.
    \item Open-source implementation available for reproducibility.
\end{itemize}

\paragraph{Paper Organization.}
Conclude the introduction with a roadmap of the paper's structure.
Section~\ref{sec:related} reviews related work.
Section~\ref{sec:conclusion} concludes with discussion of future directions.

%% ============================================
\section{Related Work}
\label{sec:related}

This section surveys relevant prior work and positions your contribution within the broader research landscape.
Organize related work thematically rather than chronologically for better clarity.

\paragraph{Traditional Approaches.}
Early work in this area focused on foundational techniques~\cite{Russell2021}.
These methods established important baselines but faced limitations in scalability and generalization.

\paragraph{Modern Deep Learning Methods.}
Recent neural network approaches~\cite{Chen2024,Smith2023} have demonstrated impressive results on benchmark tasks.
However, they typically require extensive computational resources and large datasets.

\paragraph{Efficient Training Techniques.}
Several recent works have explored optimization strategies for reducing training costs~\cite{Kumar2023}.
Our approach differs by incorporating adaptive mechanisms that dynamically adjust computational budgets.

\paragraph{Positioning Our Work.}
Unlike previous methods that focus solely on algorithmic improvements, we provide both theoretical guarantees and practical implementations, making our approach accessible to practitioners.

%% ============================================
\section{Conclusion}
\label{sec:conclusion}

This section summarizes the key contributions and discusses future directions.

We presented a novel approach to efficient neural network training that achieves significant computational savings while maintaining competitive accuracy.
Our theoretical analysis provides convergence guarantees, and empirical results on benchmark datasets validate the effectiveness of our method.

\paragraph{Future Directions.}
Several promising avenues for future work include:
(1) extending our approach to other neural architectures beyond feedforward networks,
(2) investigating applications to other domains such as natural language processing, and
(3) developing automated methods for hyperparameter selection.

\paragraph{Broader Impact.}
By reducing the computational requirements for training, our work has the potential to democratize access to machine learning tools and reduce the environmental impact of AI research~\cite{OpenAI2023}.

%% ============================================
%% ACKNOWLEDGMENTS
%% ============================================

\begin{acks}
We thank the anonymous reviewers for their helpful feedback.
This work was supported by Example Funding Agency grant number 12345.
We also thank colleagues at \OurLab{} for valuable discussions.
\end{acks}

%% ============================================
%% BIBLIOGRAPHY
%% ============================================

% Use the bibliography file in bib/references.bib
% Adjust the bibliography style as needed for your conference/journal
\bibliographystyle{ACM-Reference-Format}
\bibliography{references}

%% ============================================
%% APPENDIX (Optional)
%% ============================================

% Uncomment to add appendices
% \clearpage
% \appendix
%
% \section{Additional Experimental Results}
% \label{app:experiments}
%
% Additional tables, figures, proofs, or implementation details can go here.

\end{document}
